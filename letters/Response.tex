%\documentclass[prl,aps,preprint,showpacs,floatfix]{revtex4}
\documentclass[prl,aps,twocolumn,showpacs,floatfix]{revtex4}
\usepackage{graphicx}

\begin{document}
%\title{Response to Comment on "Non-monotonicity in the quantum-classical 
%transition: Chaos induced by quantum effects"} 
%\author{Kyle Kingsbury$^(a)$, Christopher A. Amey$^(a)$, Arie
%Kapulkin$^{(b)}$, and Arjendu K. Pattanayak$^{(a)}$} 
%\date{June 2007}
%\affiliation
%{
%} 
%\begin{abstract}
%\end{abstract}
%\pacs{PACS numbers: 05.45.Mt,03.65.Sq}
%\maketitle

The work by Finn et al further investigating our system is indeed relevant. 
They calculate maximum Lyapunov exponents ($\lambda_m$) showing 
monotonic suppression of chaos as $\beta$ is increased for $\Gamma = 0.125$, 
and no chaos for $\Gamma=0.3$.  Their results are extremely puzzling, 
particularly the complete absence of chaos for the $\Gamma = 0.3$ case, 
and definitely inconsistent with us. 

We worked with the in-principle experimentally accessible time-series. 
Poincar\'e sections indicate, for $\Gamma = 0.125$, a chaotic attractor 
at $\beta = 0.01$, which is altered but persists for $\beta = 0.3$ 
and disappears for $\beta = 0.1$. The power spectra for 
$\langle \hat X(t) \rangle$ agree, indicating chaos for the first two cases 
and no chaos for the last. For $\Gamma = 0.3$, we see no $\beta = 0.01$ 
chaos, an attractor for $\beta = 0.3$, which disappears for $\beta = 0.1$, 
in agreement with the power spectra. Further, the $\beta = 0.3$ results
look extremely similar for the two $\Gamma$ cases. The specific disagreement 
with Finn et al is the $\Gamma=0.3,\beta=0.3$ case, and also that their 
results are so different for the two $\beta = 0.3$ cases. 

We have since worked with the TISEAN package, using phase-space delay 
reconstruction of $\langle \hat X(t) \rangle$ to obtain $\lambda_m$, as 
shown in the attached figure. Each plot shows the average divergence of 
nearby points in the reconstructed phase space for $X$, on a log scale. 
Exponential growth appears as linearity before the trajectories reach 
saturation; the slope is proportional to $\lambda_m$. The delay embedding 
dimension is indicated by $m$. For each dimension, several graphs are 
shown, corresponding to different neighborhood 
choices in the reconstructed phase space. We see qualitative agreement 
with our previous results. Because of the form of the sampling in TISEAN, 
a raw slope from these figures needs to be divided by (interval * modulo) = 
$(0.01 * 11)$ to find $\lambda_m$. We estimate $\lambda_m$ for respective 
$(\Gamma, \beta)$ pairs (approximately, since they derive
from finding the slope of the straight line parts of these curves) as:
$(0.125,0.01) = 0.1, (0.125,0.3) = 0.16, (0.125,1.0) < 0.05,
(0.3,0.01) < 0.03, (0.3,0.3) = 0.13, (0.3,1.0) < 0.05$. In short,
this agrees with our previous conclusions about where chaos exists. 
Interestingly, using $\lambda_m$, the transition from quantum to
classical behavior appears to be non-monotonic for BOTH instances of
$\Gamma$.

\begin{figure}[htbp]
\centerline{\includegraphics[width=8.3cm,height=10.8cm,clip]
{TimeSeriesLyapunov.eps}}
\caption{
}
\label{figsix}
\end{figure}

Our three methods of analysis (Poincar\'e sections, power spectra, and 
time-series Lyapunov exponents) are all consistent with each other, and 
consistent with our physical understanding of how the chaos emerges 
and/or is swamped by quantum effects, and Finn et al's calculation is 
inconsistent with this for the one 'mesoscopic' case of 
$(\Gamma,\beta) =(0.3,0.3)$. The difference is intriguing. We expect that 
understanding the source of this difference -- provided it is not due 
to technical errors -- will help us in understanding something deeper about 
the physics, or about the methods of analysis.  Behind the immediate questions 
about the behavior of this model system stands the larger and fundamental 
question of whether quantum corrections always regularize and suppress 
chaotic dynamics. We believe that this, while often true, is not universal. 
For the QSD equations (or equivalent stochastic Schrodinger equations), 
that such a highly nonlinear equation has a priori a monotonic parameter 
landscape is extremely unlikely. Our perspective is supported by 
Bhattacharyya et al [our ref 14] showing that the diffusion rate for the
kicked rotor (a measure of the chaos in the system) can increase into
the quantum regime. It is only a matter of more systematic investigation
to find other such counter-examples to the folklore.

Kyle Kingsbury$^{(a)}$

Chris Amey$^{(a)}$

Arie Kapulkin$^{(a)}$

Arjendu Pattanayak$^{(a)}$

{(a) Department of Physics and Astronomy, \\
Carleton College, Northfield, Minnesota 55057
\\
(b) 128 Rockwood Cr, Thornhill, Ont L4J 7W1 Canada}

\begin{thebibliography}{99}

\bibitem{brun} T.A. Brun, I.C. Percival, and R. Schack, J. Phys. A {\bf
29}, 2077 (1996).
\bibitem{salman} T. Bhattacharya, S. Habib, and K. Jacobs, \prl {\bf
85}, 4852 (2000); S. Habib, K. Jacobs and K. Shizume, 
\prl {\bf 96}, 010403 (2006).
\end{thebibliography}
\end{document}
